\begin{problem}{Транспортировка кошек}{stdin}{stdout}{2 секунды}{256 мегабайт}

Zxr960115 содержит большое хозяйство. Он кормит $m$ милых кошечек и держит у себя $p$ кормильщиков. Через ферму проходит прямая дорога, а вдоль дороги расположено $n$ холмов, пронумерованных от 1 до $n$, слева направо. Расстояние от холма $i$ до $i-1$ равняется $d_i$ метров. Кормильщики живут на холме 1.

Однажды кошечкам захотелось порезвиться и они разбежались. Кошка $i$ пошла к холму $h_i$, дошла до него в момент времени $t_i$, а затем стала ждать кормильщика на холме $h_i$. Кормильщики должны собрать всех разбежавшихся кошек. Каждый кормильщик идет прямо от холма номер 1 до холма номер $n$, не останавливаясь у какого-либо холма, и собирает всех кошек, \textbf{ожидающих} на каждом холме. Кормильщики двигаются со скоростью 1 в единицу времени и достаточно сильны, чтобы собрать сколько угодно кошек.

Например, пусть имеется два холма $(d_2 = 1)$ и одна кошечка, которая дошла до холма 2 $(h_1 = 2)$ в момент времени 3. Тогда, если кормильщик отправится за кошками от холма 1 в момент времени 2 или 3, то он сможет забрать эту кошку. Но если он отправится от холма 1 в момент времени 1, то он не сможет этого сделать. Если кормильщик отправится за кошкой в момент времени 2, то кошка будет ждать его 0 единиц времени, если же он отправится в момент времени 3, то кошка будет ждать его 1 единицу времени.

Ваша задача --- составить расписание отправки от холма 1 для кормильщиков так, чтобы общее время ожидания кошек до того как их заберут было минимальным.

\InputFile
В первой строке входных данных содержится три целых числа $n, m, p$ $(2 \leq n \leq 10^5, 1 \leq m \leq 10^5, 1 \leq p \leq 100)$.

Во второй строке содержится $n-1$ положительных целых чисел $d_2, d_3, \dots, d_n$ $(1 \le d_i < 10^4)$.

В каждой из следующих $m$ строк содержится по два целых числа $h_i$ и $t_i$ $(1\leqh_i\leqn, 0\leqt_i\leq10^9)$.


\OutputFile
Выведите целое число, минимальную сумму времен ожидания всех кошек.

Пожалуйста, не используйте спецификатор \texttt{\%lld} для чтения и записи 64-битных чисел на С++. Рекомендуется использовать потоки \texttt{cin}, \texttt{cout} или спецификатор \texttt{\%I64d}.


\Examples

\begin{example}
\exmp{4 6 2
1 3 5
1 0
2 1
4 9
1 10
2 10
3 12
}{3
}%
\end{example}

\end{problem}
